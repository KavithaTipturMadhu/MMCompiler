
%% bare_conf.tex
%% V1.3
%% 2007/01/11
%% by Michael Shell
%% See:
%% http://www.michaelshell.org/
%% for current contact information.
%%
%% This is a skeleton file demonstrating the use of IEEEtran.cls
%% (requires IEEEtran.cls version 1.7 or later) with an IEEE conference paper.
%%
%% Support sites:
%% http://www.michaelshell.org/tex/ieeetran/
%% http://www.ctan.org/tex-archive/macros/latex/contrib/IEEEtran/
%% and
%% http://www.ieee.org/

%%*************************************************************************
%% Legal Notice:
%% This code is offered as-is without any warranty either expressed or
%% implied; without even the implied warranty of MERCHANTABILITY or
%% FITNESS FOR A PARTICULAR PURPOSE! 
%% User assumes all risk.
%% In no event shall IEEE or any contributor to this code be liable for
%% any damages or losses, including, but not limited to, incidental,
%% consequential, or any other damages, resulting from the use or misuse
%% of any information contained here.
%%
%% All comments are the opinions of their respective authors and are not
%% necessarily endorsed by the IEEE.
%%
%% This work is distributed under the LaTeX Project Public License (LPPL)
%% ( http://www.latex-project.org/ ) version 1.3, and may be freely used,
%% distributed and modified. A copy of the LPPL, version 1.3, is included
%% in the base LaTeX documentation of all distributions of LaTeX released
%% 2003/12/01 or later.
%% Retain all contribution notices and credits.
%% ** Modified files should be clearly indicated as such, including  **
%% ** renaming them and changing author support contact information. **
%%
%% File list of work: IEEEtran.cls, IEEEtran_HOWTO.pdf, bare_adv.tex,
%%                    bare_conf.tex, bare_jrnl.tex, bare_jrnl_compsoc.tex
%%*************************************************************************

% *** Authors should verify (and, if needed, correct) their LaTeX system  ***
% *** with the testflow diagnostic prior to trusting their LaTeX platform ***
% *** with production work. IEEE's font choices can trigger bugs that do  ***
% *** not appear when using other class files.                            ***
% The testflow support page is at:
% http://www.michaelshell.org/tex/testflow/



% Note that the a4paper option is mainly intended so that authors in
% countries using A4 can easily print to A4 and see how their papers will
% look in print - the typesetting of the document will not typically be
% affected with changes in paper size (but the bottom and side margins will).
% Use the testflow package mentioned above to verify correct handling of
% both paper sizes by the user's LaTeX system.
%
% Also note that the "draftcls" or "draftclsnofoot", not "draft", option
% should be used if it is desired that the figures are to be displayed in
% draft mode.
%
\documentclass[conference,10pt]{IEEEtran}
% Add the compsoc option for Computer Society conferences.
%
% If IEEEtran.cls has not been installed into the LaTeX system files,
% manually specify the path to it like:
% \documentclass[conference]{../sty/IEEEtran}





% Some very useful LaTeX packages include:
% (uncomment the ones you want to load)


% *** MISC UTILITY PACKAGES ***
%
%\usepackage{ifpdf}
% Heiko Oberdiek's ifpdf.sty is very useful if you need conditional
% compilation based on whether the output is pdf or dvi.
% usage:
% \ifpdf
%   % pdf code
% \else
%   % dvi code
% \fi
% The latest version of ifpdf.sty can be obtained from:
% http://www.ctan.org/tex-archive/macros/latex/contrib/oberdiek/
% Also, note that IEEEtran.cls V1.7 and later provides a builtin
% \ifCLASSINFOpdf conditional that works the same way.
% When switching from latex to pdflatex and vice-versa, the compiler may
% have to be run twice to clear warning/error messages.






% *** CITATION PACKAGES ***
%
%\usepackage{cite}
% cite.sty was written by Donald Arseneau
% V1.6 and later of IEEEtran pre-defines the format of the cite.sty package
% \cite{} output to follow that of IEEE. Loading the cite package will
% result in citation numbers being automatically sorted and properly
% "compressed/ranged". e.g., [1], [9], [2], [7], [5], [6] without using
% cite.sty will become [1], [2], [5]--[7], [9] using cite.sty. cite.sty's
% \cite will automatically add leading space, if needed. Use cite.sty's
% noadjust option (cite.sty V3.8 and later) if you want to turn this off.
% cite.sty is already installed on most LaTeX systems. Be sure and use
% version 4.0 (2003-05-27) and later if using hyperref.sty. cite.sty does
% not currently provide for hyperlinked citations.
% The latest version can be obtained at:
% http://www.ctan.org/tex-archive/macros/latex/contrib/cite/
% The documentation is contained in the cite.sty file itself.






% *** GRAPHICS RELATED PACKAGES ***
%
\ifCLASSINFOpdf
  % \usepackage[pdftex]{graphicx}
  % declare the path(s) where your graphic files are
  % \graphicspath{{../pdf/}{../jpeg/}}
  % and their extensions so you won't have to specify these with
  % every instance of \includegraphics
  % \DeclareGraphicsExtensions{.pdf,.jpeg,.png}
\else
  % or other class option (dvipsone, dvipdf, if not using dvips). graphicx
  % will default to the driver specified in the system graphics.cfg if no
  % driver is specified.
  % \usepackage[dvips]{graphicx}
  % declare the path(s) where your graphic files are
  % \graphicspath{{../eps/}}
  % and their extensions so you won't have to specify these with
  % every instance of \includegraphics
  % \DeclareGraphicsExtensions{.eps}
\fi
% graphicx was written by David Carlisle and Sebastian Rahtz. It is
% required if you want graphics, photos, etc. graphicx.sty is already
% installed on most LaTeX systems. The latest version and documentation can
% be obtained at: 
% http://www.ctan.org/tex-archive/macros/latex/required/graphics/
% Another good source of documentation is "Using Imported Graphics in
% LaTeX2e" by Keith Reckdahl which can be found as epslatex.ps or
% epslatex.pdf at: http://www.ctan.org/tex-archive/info/
%
% latex, and pdflatex in dvi mode, support graphics in encapsulated
% postscript (.eps) format. pdflatex in pdf mode supports graphics
% in .pdf, .jpeg, .png and .mps (metapost) formats. Users should ensure
% that all non-photo figures use a vector format (.eps, .pdf, .mps) and
% not a bitmapped formats (.jpeg, .png). IEEE frowns on bitmapped formats
% which can result in "jaggedy"/blurry rendering of lines and letters as
% well as large increases in file sizes.
%
% You can find documentation about the pdfTeX application at:
% http://www.tug.org/applications/pdftex





% *** MATH PACKAGES ***
%
%\usepackage[cmex10]{amsmath}
% A popular package from the American Mathematical Society that provides
% many useful and powerful commands for dealing with mathematics. If using
% it, be sure to load this package with the cmex10 option to ensure that
% only type 1 fonts will utilized at all point sizes. Without this option,
% it is possible that some math symbols, particularly those within
% footnotes, will be rendered in bitmap form which will result in a
% document that can not be IEEE Xplore compliant!
%
% Also, note that the amsmath package sets \interdisplaylinepenalty to 10000
% thus preventing page breaks from occurring within multiline equations. Use:
%\interdisplaylinepenalty=2500
% after loading amsmath to restore such page breaks as IEEEtran.cls normally
% does. amsmath.sty is already installed on most LaTeX systems. The latest
% version and documentation can be obtained at:
% http://www.ctan.org/tex-archive/macros/latex/required/amslatex/math/





% *** SPECIALIZED LIST PACKAGES ***
%
%\usepackage{algorithmic}
% algorithmic.sty was written by Peter Williams and Rogerio Brito.
% This package provides an algorithmic environment fo describing algorithms.
% You can use the algorithmic environment in-text or within a figure
% environment to provide for a floating algorithm. Do NOT use the algorithm
% floating environment provided by algorithm.sty (by the same authors) or
% algorithm2e.sty (by Christophe Fiorio) as IEEE does not use dedicated
% algorithm float types and packages that provide these will not provide
% correct IEEE style captions. The latest version and documentation of
% algorithmic.sty can be obtained at:
% http://www.ctan.org/tex-archive/macros/latex/contrib/algorithms/
% There is also a support site at:
% http://algorithms.berlios.de/index.html
% Also of interest may be the (relatively newer and more customizable)
% algorithmicx.sty package by Szasz Janos:
% http://www.ctan.org/tex-archive/macros/latex/contrib/algorithmicx/




% *** ALIGNMENT PACKAGES ***
%
%\usepackage{array}
% Frank Mittelbach's and David Carlisle's array.sty patches and improves
% the standard LaTeX2e array and tabular environments to provide better
% appearance and additional user controls. As the default LaTeX2e table
% generation code is lacking to the point of almost being broken with
% respect to the quality of the end results, all users are strongly
% advised to use an enhanced (at the very least that provided by array.sty)
% set of table tools. array.sty is already installed on most systems. The
% latest version and documentation can be obtained at:
% http://www.ctan.org/tex-archive/macros/latex/required/tools/


%\usepackage{mdwmath}
%\usepackage{mdwtab}
% Also highly recommended is Mark Wooding's extremely powerful MDW tools,
% especially mdwmath.sty and mdwtab.sty which are used to format equations
% and tables, respectively. The MDWtools set is already installed on most
% LaTeX systems. The lastest version and documentation is available at:
% http://www.ctan.org/tex-archive/macros/latex/contrib/mdwtools/


% IEEEtran contains the IEEEeqnarray family of commands that can be used to
% generate multiline equations as well as matrices, tables, etc., of high
% quality.


%\usepackage{eqparbox}
% Also of notable interest is Scott Pakin's eqparbox package for creating
% (automatically sized) equal width boxes - aka "natural width parboxes".
% Available at:
% http://www.ctan.org/tex-archive/macros/latex/contrib/eqparbox/





% *** SUBFIGURE PACKAGES ***
%\usepackage[tight,footnotesize]{subfigure}
% subfigure.sty was written by Steven Douglas Cochran. This package makes it
% easy to put subfigures in your figures. e.g., "Figure 1a and 1b". For IEEE
% work, it is a good idea to load it with the tight package option to reduce
% the amount of white space around the subfigures. subfigure.sty is already
% installed on most LaTeX systems. The latest version and documentation can
% be obtained at:
% http://www.ctan.org/tex-archive/obsolete/macros/latex/contrib/subfigure/
% subfigure.sty has been superceeded by subfig.sty.



%\usepackage[caption=false]{caption}
%\usepackage[font=footnotesize]{subfig}
% subfig.sty, also written by Steven Douglas Cochran, is the modern
% replacement for subfigure.sty. However, subfig.sty requires and
% automatically loads Axel Sommerfeldt's caption.sty which will override
% IEEEtran.cls handling of captions and this will result in nonIEEE style
% figure/table captions. To prevent this problem, be sure and preload
% caption.sty with its "caption=false" package option. This is will preserve
% IEEEtran.cls handing of captions. Version 1.3 (2005/06/28) and later 
% (recommended due to many improvements over 1.2) of subfig.sty supports
% the caption=false option directly:
%\usepackage[caption=false,font=footnotesize]{subfig}
%
% The latest version and documentation can be obtained at:
% http://www.ctan.org/tex-archive/macros/latex/contrib/subfig/
% The latest version and documentation of caption.sty can be obtained at:
% http://www.ctan.org/tex-archive/macros/latex/contrib/caption/




% *** FLOAT PACKAGES ***
%
\usepackage{fixltx2e}
% fixltx2e, the successor to the earlier fix2col.sty, was written by
% Frank Mittelbach and David Carlisle. This package corrects a few problems
% in the LaTeX2e kernel, the most notable of which is that in current
% LaTeX2e releases, the ordering of single and double column floats is not
% guaranteed to be preserved. Thus, an unpatched LaTeX2e can allow a
% single column figure to be placed prior to an earlier double column
% figure. The latest version and documentation can be found at:
% http://www.ctan.org/tex-archive/macros/latex/base/



%\usepackage{stfloats}
% stfloats.sty was written by Sigitas Tolusis. This package gives LaTeX2e
% the ability to do double column floats at the bottom of the page as well
% as the top. (e.g., "\begin{figure*}[!b]" is not normally possible in
% LaTeX2e). It also provides a command:
%\fnbelowfloat
% to enable the placement of footnotes below bottom floats (the standard
% LaTeX2e kernel puts them above bottom floats). This is an invasive package
% which rewrites many portions of the LaTeX2e float routines. It may not work
% with other packages that modify the LaTeX2e float routines. The latest
% version and documentation can be obtained at:
% http://www.ctan.org/tex-archive/macros/latex/contrib/sttools/
% Documentation is contained in the stfloats.sty comments as well as in the
% presfull.pdf file. Do not use the stfloats baselinefloat ability as IEEE
% does not allow \baselineskip to stretch. Authors submitting work to the
% IEEE should note that IEEE rarely uses double column equations and
% that authors should try to avoid such use. Do not be tempted to use the
% cuted.sty or midfloat.sty packages (also by Sigitas Tolusis) as IEEE does
% not format its papers in such ways.





% *** PDF, URL AND HYPERLINK PACKAGES ***
%
\usepackage{url}
% url.sty was written by Donald Arseneau. It provides better support for
% handling and breaking URLs. url.sty is already installed on most LaTeX
% systems. The latest version can be obtained at:
% http://www.ctan.org/tex-archive/macros/latex/contrib/misc/
% Read the url.sty source comments for usage information. Basically,
% \url{my_url_here}.





% *** Do not adjust lengths that control margins, column widths, etc. ***
% *** Do not use packages that alter fonts (such as pslatex).         ***
% There should be no need to do such things with IEEEtran.cls V1.6 and later.
% (Unless specifically asked to do so by the journal or conference you plan
% to submit to, of course. )
\usepackage{amssymb}
\usepackage{graphicx}
\usepackage{amsmath}
\usepackage{algorithm}
\usepackage{algorithmicx}
\usepackage{algcompatible}
\usepackage{subfloat}
% correct bad hyphenation here
\hyphenation{op-tical net-works semi-conduc-tor}


\begin{document}
%
% paper title
% can use linebreaks \\ within to get better formatting as desired
\title{LLVM-Based IR for Distributed Dynamic Dataflow Runtime}
%TODO: Title to capture the scheduling etc. Not the datapath.

% author names and affiliations
% use a multiple column layout for up to three different
% affiliations
%\author{\IEEEauthorblockN{Kavitha Madhu, Saptarsi Das, Madhava Krishna, Nalesh S, S. K. Nandy, Ranjani Narayan}
%\IEEEauthorblockA{Indian Institute of Science, Morphing Machines\\
% Georgia Institute of Technology\\
% Atlanta, Georgia 30332--0250\\
%Email: kavitha, sdas, madhav, nalesh@cadl.iisc.ernet.in, nandy@serc.iisc.in, ranjani.narayan@morphingmachines.com\\
 %\and
 %\IEEEauthorblockN{}
 %\IEEEauthorblockA{Morphing Machines\\
% Springfield, USA\\
% Email: homer@thesimpsons.com}
% \and
% \IEEEauthorblockN{James Kirk\\ and Montgomery Scott}
% \IEEEauthorblockA{Starfleet Academy\\
% San Francisco, California 96678-2391\\
% Telephone: (800) 555--1212\\
%Fax: (888) 555--1212
%}}

% conference papers do not typically use \thanks and this command
% is locked out in conference mode. If really needed, such as for
% the acknowledgment of grants, issue a \IEEEoverridecommandlockouts
% after \documentclass

% for over three affiliations, or if they all won't fit within the width
% of the page, use this alternative format:
% 
% \author{\IEEEauthorblockN{Kavitha T Madhu\IEEEauthorrefmark{1}, S. K. Nandy\IEEEauthorrefmark{1} and
% Ranjani Narayan\IEEEauthorrefmark{2},
% }
% \IEEEauthorblockA{\IEEEauthorrefmark{1}Indian Institute of Science, Bangalore, India\\
% Email: \{kavitha\}@cadl.iisc.ernet.in, nandy@cds.iisc.ac.in}
% \IEEEauthorblockA{\IEEEauthorrefmark{2}Morphing Machines Pvt. Ltd. , Bangalore, India\\
% Email: ranjani.narayan@morphingmachines.com}}





% use for special paper notices
%\IEEEspecialpapernotice{(Invited Paper)}




% make the title area
\maketitle


\begin{abstract}
%\boldmath
In this paper we present extensions to LLVM IR to represent task parallelism in applications in a target agnostic manner.  The parallel IR uses metadata to represent the task-interaction graph of an application, much like DFGR \ref{}, enabling macro-level optimizations such as inter-task communication optimizations. It also offers the ability to express data distribution patterns, similarly as PGAS languages such as Chapel. We present an interface to plug in libraries that present parallel patterns as well as data distribution patterns, thus easing the use of parallel languages to target the underlying runtime. Effectiveness of the proposed IR is evaluated on the REDEFINE CGRA which has a dynamic dataflow runtime model with distributed memory.
\end{abstract}
% IEEEtran.cls defaults to using nonbold math in the Abstract.
% This preserves the distinction between vectors and scalars. However,
% if the conference you are submitting to favors bold math in the abstract,
% then you can use LaTeX's standard command \boldmath at the very start
% of the abstract to achieve this. Many IEEE journals/conferences frown on
% math in the abstract anyway.

% no keywords




% For peer review papers, you can put extra information on the cover
% page as needed:
% \ifCLASSOPTIONpeerreview
% \begin{center} \bfseries EDICS Category: 3-BBND \end{center}
% \fi
%
% For peerreview papers, this IEEEtran command inserts a page break and
% creates the second title. It will be ignored for other modes.
\IEEEpeerreviewmaketitle


\renewcommand{\algorithmicrequire}{\textbf{Input:}}
\renewcommand{\algorithmicensure}{\textbf{Output:}}
\section{Introduction}
\label{sec_intro}
In order to achieve optimal performance, it is important to capture the intents of an algorithm in a program that fully exploits the underlying processing platform. Several programming languages approach the problem of expressing parallel algorithms via support for multithreading, multiprocessing etc. Programming models such as OpenMP and MPI provide extensions to existing languages to support parallel patterns, the newer versions of which support arbitrary task graphs (OpenMP 3.0\ref{}). PGAS programming languages such as X10\ref{}, Chapel \ref{} ease the programmers' burden by offering language constructs that spawn tasks and manage inter-task synchronization, also allowing the programmer to distribute data onto physically distributed memory with a  shared address space. Compiling such languages onto the same target runtime can be a challenge, as is reusing high level optimizations due to custom compilation toolchains. There hence arises the need for a parallel IR which allows compiling various applications easily onto the same target, while enabling common optimizations among them. CnC is one such language designed to increase programmability that separates the concerns of expressing parallelism and tuning it to specific targets. Further, DFGR was proposed as an intermediate representation that assumes data race free execution of tasks. It extends CnC by supporting direct predication of a task by another and allowing communication of aggregate data objects such as ranges and regions between tasks. 

In this work, we present an intermediate representation based on DFGR implemented as an extension of LLVM IR which eases static analysis and optimizations of applications at a macro level. The proposed IR captures the dynamic behavior of tasks that compose the application and further ensures deadlock freedom in dependences across tasks, as will be explained subsequently. The IR also comprises a subset of tuning directives to distribute input and output data onto various chunks of memory by allowing domain map definitions. Further, data parallelism is enabled exposing the underlying hardware resources as locales, an idea borrowed from Chapel programming language, similarly as \textit{places} in X10. A subset of regularly occurring parallel patterns and data distribution libraries are provided that maybe enumerated into native graph description, enabling the compiler to perform target specific optimizations, such as offering locality in activation frames. 
\section{Description of Task-Interaction Graph}
\label{sec_ir}
\label{sec:llvm_ir}
LLVM IR generates code using its virtual instruction set targeting unicore processors, making it unsuitable for representing parallel programs. State of the art parallel frontends are usually designed to suit the underlying hardware model (shared memory multithreading model assumed by OpenMP vs distributed memory model in case of MPI). A target agnostic representation of parallel programs based on DFGR\cite{} is presented in this work that describes kernels in a graph-like manner. An application kernel described using DFGR comprises compute steps that represent scheduleable computations and item collections that represent data communicated between them. Compute step instances maybe spawned by other compute steps and the creator-created, controller-controllee relationships are captured in DFGR. This representation aims at easing the process of developing a new frontend for any parallel hardware target so long as there exists runtime support to execute the kernel on the target. DFGR also encompasses dynamic behavior of applications unlike the dataflow graph representation.

We propose realization of DFGR (in the current LLVM framework) by a parallel IR whose description is as follows.
\subsection{Representing Computation}
Compute steps of DFGR are represented as functions in LLVM IR. Statically enumerated compute steps are uniquely identified by the function's signature as the identifier. Dynamic instances of compute steps are identified by the function signature along with a tag, akin to the DFGR compute step tags. The tag is used to derive producer-consumer relationship between compute steps in our work and does not see a direct realization at runtime. To ease static analysis, we use limited set of functions defined on tags described next to define relationships between dynamic compute step instances. 

Operands of a task are represented as arguments to the function. In a REDEFINE-esque landscape where a subset of arguments of a HyperOp maybe passed through a context frame, the arguments need to be pinned to registers, similarly as the GHC\ref{} by defining a custom calling convention. The choice of the set of arguments that need to be passed through context frames is a decision that the compiler makes and it is indicated by marking the arguments as \textit{inreg}, implying that the arguments are passed in registers.
\begin{itemize}
 \item $<id>$ is a self-referential tag of a compute step.
 \item $<suffix(id,i)>$ represents a step with suffix \textit{i} of the self referential id. 
 \item $<prefix(id,n)>$ represents the prefix of an id obtained by considering the first n entries of the self-referential id. Further, nesting of prefix function is allowed.
\end{itemize}

We observe that the aforementioned functions suffice for representing any kernel in our IR with the assumption that runtime manages instances independent of tag functions. Arbitrary tag functions make static analysis difficult but maybe easier to program. Hence, we support linear tag relationships to ease the programmer's burden. Any linear transformation of the form $a\times id + b $ is supported, where $a$ and $b$ are scalars and so is further nesting of tag functions (for example, transformation of the form $a\times prefix(id, i) + b $).

\subsection{Representing Communication}
Communication of data between compute steps of the parallel IR is represented using custom named metadata in LLVM IR that labels data with its consumer(s). Approaches proposed in \cite{} realize communication via intrinsic extensions to LLVM and assume software support for communication. On the other hand, our approach makes no assumptions about the mode of realizing communication and hence can be compiled to esoteric targets(ASIC, FPGA) as well without performing machine level optimizations of removing communication instructions. The parallel IR defined as per the specification listed above can be compiled suitably onto any target and is open to local as well as graph optimizations and modifies the source IR to the least extent possible. 

Communication in our parallel IR is asynchronous and the consumer is unaware of the source of received data. Communication between compute steps i.e., HyperOps are of three types namely \textit{data, control, synchronization}. Interim data produced by a HyperOp (represented as a function) is annotated to define who the consumers are and what the data is used for (i.e., whether \textit{ConsumedBy} or \textit{Controls} or \textit{SyncWith} consumer HyperOp(s)). In order to offer the flexibility of either using a single copy of the data that gets consumed by other HyperOps or to follow dynamic single assignment rule, data is not associated with instance ids (tags). This deviates slightly from DFGR where data is associated with instance ids uniformly. DFGR also assumes that compute steps are side-effect free. In our case, compute step instances that modify the global state are explicitly serialized by the compiler. Further, the data being communicated is marked as \textit{static} or \textit{composite} in order to assist the compiler in mapping data onto different modes of communication efficiently. Additionally, in case of the other two forms of communication i.e., \textit{control} and \textit{synchronization}, the runtime is free to realize data-based synchronization and control flow (like CnC) or may choose to support explicit control and synchronization primitives. The consumer HyperOps, when statically enumerated, are listed with the metadata linked with the data. A range of consumers may also be specified by extending the suffix function (described in the previous section) with vector extensions. Recall that $<suffix(id,i)>$ was earlier defined over scalar ids. These functions are extended to to support vector inputs in place of $i$ which can now take a range of integer values specified as $[m:n]$, including $m$ and $n$. $n$ maybe variable but the lower bound $m$ needs to be known. As mentioned previously, linear transformation defined over a range may also be used as the vector input to suffix function.

\textbf{TODO: Topics of discussion here:\\
Disadvantages of the current IR implementation: Traditional optimizations may not be easily performed on the IR since metadata is ignored during compiler optimizations. CnC defines producer-consumer relationships at the boundaries of HyperOps and their compiler generates skeleton code for the programmer to fill with \textit{get} and \textit{put} calls assuming item collections to exist. This allows the communication to have a memory footprint and hence allows optimizations to go on since the memory operations change the global state. However, this hinders graph optimizations such as locality optimizations. Another approach proposed for PGAS languages on LLVM IR states that a shared address space in LLVM maybe written into or read from. I need to verify if a large number of address spaces can be created in LLVM and if they can be parametrically created and addressed. If this is indeed available, a pass that converts communication to writes and reads to address spaces can be implemented that allows standard optimizations to proceed. Another pass that eliminates these instructions and reintroduces the metadata maybe implemented. If feasible, this should be an ideal solution to our problems.}
\section{Tuning for the Underlying Architecture}
\label{sec_data}
In addition to expressing parallelism, a programmer should also be able to reason about performance of the application and tune it for optimality on the underlying hardware. As mentioned previously, we borrow the definition of \textit{Locales} from Chapel\cite{} programming language. Each locale represents a compute node and its associated chunk of distributed memory in the underlying hardware.
The context of hardware in which the application runs is a set of locales, either composed with affinities defined by the compiler or specified by the user. Programmer may tune the application onto the underlying hardware in the following ways.
\subsection{Locale Declaration}
\label{subsec_localeset}
A set of locales may optionally be specified by the user via a local space declaration in the IR as follows:
\begin{itemize}
 \item LocaleSet = ${1:m,1:n}$ which specifies a rectangle of compute resources of size $m\times n$.
 \item LocaleSet = ${{x1,y1},{x2,y2}..}$ where each ${xi,yi}$ specifies coordinates of the resource in the underlying mesh.
\end{itemize}
If not declared, the compiler computes LocaleSet for optimal performance. 

\subsection{Data Layout and Distribution onto Locales}
Data needs to be redistributed across Locales for locality guarantees in NUMA architectures. Explicit distribution onto locales identified via their {x,y} coordinates is useful in applications with statically enumerable graphs or certain dynamic but regular graphs or applications that don't follow known patterns across data-accesses. Data parallelism exhibited by irregular data dependent applications perform well when known distribution techniques are employed.

A multidimensional array to be distributed across the allocated locales has the following annotations:\\
\begin{itemize}
\item {Block, blocksize$={n1,n2,..}, Locale={LocaleSet Index1, LocaleSet Index2,..}/Locale={start index1:end index1, start index2:end index2}$}
\item {Cyclic}
\end{itemize}
Specifying Locales in either case is optional when distributing data since the Locales may not be known at compile time. When the distribution is defined (Cyclic/Block) without locales, an implicit iterator for LocaleSet is assumed (which increments along the column and then row when rectangular resource allocation scheme is employed or through the LocaleSet from left to right if the second declaration from \ref{subsec_localeset} is used). \textit{blocksize} need not be specified along all array dimensions.

Other than Cyclic and Block data distribution patterns, we support BlockCyclic, Random and Recursive Bisection as other patterns of distribution. \textbf{TODO: More work required in this to identify patterns and their suitability.}

\subsection{Affinity of Computation onto Locales}
Compiler defined distribution of tasks onto locales is easier to perform when the tasks are static instances, where a explicit locale is specified similarly as data layout locales. In case of dynamic instances, defining distribution of computations onto locales may either be explicit (using their {x,y} identifier) or maybe distributed cyclically where each task instance is assigned to a locale iterated over at runtime. The latter case proves useful in case of data parallel i.e., SIMD/SIMT applications when static analysis does not come in handy.
\section{Support for Parallel Compute and Data Distribution Patterns}
\label{sec_patterns}
\textbf{The plan is to support these for a parallel frontend}

Development of parallel algorithms and using them as building blocks in solving problems can be accelerated via support for parallel patterns of computation. Expressing such regular patterns embarassing parallelism is facilitated in the proposed compilation flow via support for parallel patterns in IR, which maybe generated from an explicitly parallel frontend. The parallel compute patterns are similar to macros/directives that are supported via annotations to loops representing the high level task-graph structure. The patterns come with an annotations processor interface that allows pattern specific optimizations (viz generation of equivalent task graph with larger span) to be performed and then expands the annotations onto the previously described LLVM IR format and resolves dependences between them. Supported patterns and their realization via annotations are listed in table \ref{tab_list_patterns}.
\begin{table}[htp]
\caption{Supported Parallel Compute Patterns}
\label{tab_list_patterns}
\centering
    \begin{tabular}{|p{6cm} | p{6cm}|}
    \hline 
    \textbf{Pattern} & \textbf{Annotation} \\\hline \hline
     Map, Stencil & parallel.for with blocksize (default size=1) over a loop nest \\ \hline
     Fork-join, Reduction, Scan  & parallel.for.deterministicreduction and parallel.for.reduction with predefined monotonic combiner functions (add, mul, min, max, and, or), parallel.scanreduction \\ \hline 
%      Stream &\\ \hline  
     \end{tabular}
\end{table}

% 


% An example of a floating figure using the graphicx package.
% Note that \label must occur AFTER (or within) \caption.
% For figures, \caption should occur after the \includegraphics.
% Note that IEEEtran v1.7 and later has special internal code that
% is designed to preserve the operation of \label within \caption
% even when the captionsoff option is in effect. However, because
% of issues like this, it may be the safest practice to put all your
% \label just after \caption rather than within \caption{}.
%
% Reminder: the "draftcls" or "draftclsnofoot", not "draft", class
% option should be used if it is desired that the figures are to be
% displayed while in draft mode.
%
%\begin{figure}[!t]
%\centering
%\includegraphics[width=2.5in]{myfigure}
% where an .eps filename suffix will be assumed under latex, 
% and a .pdf suffix will be assumed for pdflatex; or what has been declared
% via \DeclareGraphicsExtensions.
%\caption{Simulation Results}
%\label{fig_sim}
%\end{figure}

% Note that IEEE typically puts floats only at the top, even when this
% results in a large percentage of a column being occupied by floats.


% An example of a double column floating figure using two subfigures.
% (The subfig.sty package must be loaded for this to work.)
% The subfigure \label commands are set within each subfloat command, the
% \label for the overall figure must come after \caption.
% \hfil must be used as a separator to get equal spacing.
% The subfigure.sty package works much the same way, except \subfigure is
% used instead of \subfloat.
%
%\begin{figure*}[!t]
%\centerline{\subfloat[Case I]\includegraphics[width=2.5in]{subfigcase1}%
%\label{fig_first_case}}
%\hfil
%\subfloat[Case II]{\includegraphics[width=2.5in]{subfigcase2}%
%\label{fig_second_case}}}
%\caption{Simulation results}
%\label{fig_sim}
%\end{figure*}
%
% Note that often IEEE papers with subfigures do not employ subfigure
% captions (using the optional argument to \subfloat), but instead will
% reference/describe all of them (a), (b), etc., within the main caption.


% An example of a floating table. Note that, for IEEE style tables, the 
% \caption command should come BEFORE the table. Table text will default to
% \footnotesize as IEEE normally uses this smaller font for tables.
% The \label must come after \caption as always.
%
%\begin{table}[!t]
%% increase table row spacing, adjust to taste
%\renewcommand{\arraystretch}{1.3}
% if using array.sty, it might be a good idea to tweak the value of
% \extrarowheight as needed to properly center the text within the cells
%\caption{An Example of a Table}
%\label{table_example}
%\centering
%% Some packages, such as MDW tools, offer better commands for making tables
%% than the plain LaTeX2e tabular which is used here.
%\begin{tabular}{|c||c|}
%\hline
%One & Two\\
%\hline
%Three & Four\\
%\hline
%\end{tabular}
%\end{table}


% Note that IEEE does not put floats in the very first column - or typically
% anywhere on the first page for that matter. Also, in-text middle ("here")
% positioning is not used. Most IEEE journals/conferences use top floats
% exclusively. Note that, LaTeX2e, unlike IEEE journals/conferences, places
% footnotes above bottom floats. This can be corrected via the \fnbelowfloat
% command of the stfloats package.








% conference papers do not normally have an appendix


% use section* for acknowledgement
% 



% trigger a \newpage just before the given reference
% number - used to balance the columns on the last page
% adjust value as needed - may need to be readjusted if
% the document is modified later
%\IEEEtriggeratref{8}
% The "triggered" command can be changed if desired:
%\IEEEtriggercmd{\enlargethispage{-5in}}

% references section

% can use a bibliography generated by BibTeX as a .bbl file
% BibTeX documentation can be easily obtained at:
% http://www.ctan.org/tex-archive/biblio/bibtex/contrib/doc/
% The IEEEtran BibTeX style support page is at:
% http://www.michaelshell.org/tex/ieeetran/bibtex/
%\bibliographystyle{IEEEtran}
% argument is your BibTeX string definitions and bibliography database(s)
%\bibliography{IEEEabrv,../bib/paper}
%
% <OR> manually copy in the resultant .bbl file
% set second argument of \begin to the number of references
% (used to reserve space for the reference number labels box)
%\begin{thebibliography}{1}
%\renewcommand{\bibfont}{\tiny}
\bibliographystyle{IEEEtran}
% \bibliography{references.bib}
%\bibitem{IEEEhowto:kopka}
%H.~Kopka and P.~W. Daly, \emph{A Guide to \LaTeX}, 3rd~ed.\hskip 1em plus
%  0.5em minus 0.4em\relax Harlow, England: Addison-Wesley, 1999.

%\end{thebibliography}




% that's all folks
\end{document}


